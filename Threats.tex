\section{Threats to Validity}

While we believe that our experiment ultimately makes a valuable test of the framework which, for the scope of this project, gives an initial feel for its applicability there are a couple of possible threads to validity.

\subsection{Internal Threats}
\label{sub:Internal Threats}

\subsubsection{Translation}
\label{sub:Translation}

We are studying a project with existing contracts which we do to ensure certain diversities in the input to the framework. Using contracts written for a specific project provides a genuineness that makes sure we do not unintentionally come up with input that we know will result in a positive test. On the other hand, since the code contracts from the project is not in a suitable format we need to translate them into behaviours before we feed them to the framework.
 
This poses a possible risk that something will be “lost in translation”, especially because we are not the original authors of the contracts and therefore we cannot be sure to capture all the semantic meaning when doing the translation. The threat is intensified by the fact that the translation goes from something that is programmatic and very concise to something that is “wordy” and more high-level due to the nature of human language.
 
Arguably, the focus of this project is on the mechanical conversion between behaviours and code contracts and not the other way around so even though the threat is definitely adherent to the project it will not have a major impact. In the worst case, we will be sacrificing (some) of the genuineness that we gained from using contracts from a “real” project instead of inventing them ourselves – which we believe is okay.

\subsubsection{High-level Usefulness}
\label{sub:High-level Usefulness}

Our framework is intended as a shared tool for programmers and non-programmers, business analysts and the like to use for collaboration in software development. As such, the framework is built on what could be viewed as a three-tier architecture including code contracts, behaviours and a translation layer between them.
 
We use the Votáil project and the experienced researchers that created it to argue for the quality of the code contracts and we can argue for the quality of the translation simply by comparing the result of the conversion directly to the original contract. It seems reasonable to require that we also evaluate the suitability of the behaviours that form the input to the conversion in a less technical context – however, doing so would require some involvement from a user who (a) is a non-programmer and (b) is unfamiliar with the behaviour language and it’s implementation to act as an unbiased test-person.

Since these competencies are not available in our team and such a test is time consuming to conduct, we restrict ourselves to mention it as a potential threat to validity. Even though it is adherent as a threat to the project in an industrial context, we believe it is a minor threat in the scope of this report because the primary focus of the project is on behaviour translation and model driven development.

\subsection{External Threats}
\label{sub:External Threats}

\subsubsection{Statistical Insignificance}
\label{sub:Statistical Insignificance}

Our framework targets behaviour driven development in general regardless of the area of application. Nevertheless, our experiment builds on contracts from the Vótáil project which is specifically concerned with the Irish election system. In order to present a thoroughly tested framework it seems reasonable to include tests on contracts from multiple domains – e.g. software for elections, biology, the dental business or whatever. Otherwise we cannot claim the results of the evaluation to be statistically representative for all domains.
 
It seems even more relevant because the kind of contracts we work with involves a natural language dimension and natural language usually includes some business specific terminologies (as opposed to mathematical operators used in normal code contracts). These could easily end up in a behaviour specification (it even should as part of the intention with raising the level of abstraction is that you can use a language that is more familiar to the people working in the domain).
 
A property about our framework that keeps this threat to a minimum is that we include a simple, but flexible translation language that specifies how each behaviour is translated into code. Since each behaviour translation is specified individually we have a lot of control over the process, so it’s hard to imagine a behaviour that would not work with our framework even though it might come from another domain than the Vótáil project.
 
This argument, along with the fact that we choose, for this project, to focus on the translation lowers the intensity of the threat so much that we will not pay it a lot of attention. In an industrial context, the importance of this threat is of course significantly higher.



