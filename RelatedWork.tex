\chapter{Related Work}
How to read
\begin{itemize}
\item At this point you have read the title, abstract, intro, related work and conclusion.
\item Read the entire paper without stopping
\item Skip parts that you do not understand.
\item Most papers can be processed like that within an hour.
\item Read the paper second time. Normally you can understand a lot more.
\item Sketch an outline, mindmap for the paper.
\item Analyze claims and evidence critically.
\item You are ready to write a 3-4 lines summary for your related work section
\end{itemize}

Citing
\begin{itemize}
\item 15-30 references, the last page (more in a thesis)
\item About half is related work. Rest is background information.
\item Prioritize trustworthy papers, respected venues and authors.
\item Papers with language mistakes, unclear, badly organized are often also conceptually flawed (this is also why you should avoid these pitfalls in your writing).
\item Wikipedia entries, industrial white papers, newspaper articles, and student term projects, are most often not suitable to cite as primary sources.
\item Avoid quotes, avoid long quotes, avoid figure quotes.
\item Always cite if you quote. Also for figures! Quoting without citing is failed exam and a risk of exmatriculation.
\item All references should have uniform look (BiBTeX if possible)
\end{itemize}

Writing the section
\begin{itemize}
\item Transform your notes into a 3-4 lines summary
\item Not enough just to list.
  \begin{itemize}
  \item Bad Another related work is XXX
  \item Good XXX addresses the same problem in a different way. The advantage of our solution is that it is faster for cases when .... Also XXX provides no evidence whether the solution really scales if ...On the other hand XXX exploits YYY, which would be interesting to consider as an extension of our work.
  \end{itemize}
\item Explain precisely what are the advantages, or the nature of relation between your work and the other work.
\item Section can take anything from 3 pages to3/4 of a page
\item Can delegate writing the section and reading related work.
\item All can be examined in the contents of this section.
\end{itemize}