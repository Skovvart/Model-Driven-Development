%% Argument why each different DSL is advantageous
\subsection{Extending Gherkin to Support Contracted Software}
\label{sub:Extending Gherkin to Support Contracted Software}

The specification of informal behaviors in our framework is done using an
extended version of the Gherkin description language.
The extensions made allow for the gradual conversion between classical step-oriented behaviors to generalized behaviors, such that they can later be translated to contracts.

%%% Removed BDD description, moved to Background //Søren

\subsubsection{Generalizing Behaviors}
\label{sub:Generalizing Behaviors}

Our primary extensions to Gherkin are the addition of general
behavior clauses. A general behavior corresponds to a subroutine
and consists of a series of requirement blocks and ensuring blocks
(see Listing~\ref{lst:featureexample}).

\begin{lstlisting}[caption={General Behavior Description of Adding Natural Numbers},label={lst:featureexample}]
  Feature: ...
  Behavior: Add two numbers
     Requiring <input_1> is a positive number
     And <input_2> is a positive number
     Ensuring the result is the sum of <input_1> and <input_2>
     And result is a positive number
\end{lstlisting}


These preconditions and postconditions are written completely in
natural language, and informally specifies which contracts the corresponding
behavior has.

\subsubsection{Keeping the Spirit of Behavior Driven Development}
\label{ssub:Keeping the Spirit of Behavior Driven Development}
By extending the Gherkin language with only a few keywords, we preserve the flexibility of BDD and keep the same level of expressiveness that can be expected. In the end it is also possible to regenerate features without
the contract-specific parts, and thus allow them to be served as input for existing tools.


\subsection{Transforming Behaviors to Contracts using Lettuce}
\label{sub:Transforming Behaviors to Contracts using Lettuce}

In order to translate high-level informal behaviors to low-level formal
contracts we have created a small DSL called Lettuce.
Lettuce allows direct translation of behaviors to methods,
and rule-wise pattern based translation for translating contract expressions.

\subsubsection{Mapping Behaviors}
\label{sub:Mapping Behaviors}

Behaviors from Gherkin are mapped to a method using a similarly named
construct in Lettuce. Behaviors, in Lettuce, need a specification which
description matches the description of the feature definitions (see Listing~\ref{lst:featureexample}).
Furthermore, a behavior needs to know which class it belongs to, and how the actual method signature should look
(see Listing~\ref{lst:behtransexample}).

\begin{lstlisting}[caption={General Behavior Description of Adding Natural Numbers},label={lst:behtransexample}]
  behavior 'Add two numbers'
    class Calculator
    signature 'public int add(int input1, int input2)'
  end
\end{lstlisting}

The actual translation process is rather simple. Each behavior from the feature definition is mapped to the
corresponding class with the specified signature. Then each of its pre- 
and postconditions are mapped using the defined rules and attached to the method.

\subsubsection{Rewriting Sentences to Logical Expressions}
\label{sub:RewritingSentencestoLogicalExpressions}

Pre- and postconditions are similar in the way they are constructed, because they both rely on boolean expressions.
Therefore, we constructed our transformation language so that it uses the same transformation rules to translate
boolean expressions. The only difference is that preconditions translate to requirement blocks and postconditions translate to 
ensuring blocks --- in that way, we save the user the trouble of having to write a lot duplicated transformations.

\begin{lstlisting}[caption={Different Types of Transformation Rules in Lettuce},label={lst:ruleexample}]
  rule positive $input is a positive number
    match greater_than $x=$input $y='0'
  end

  rule greater_than $x is greater than $y
    expr '$x > $y'
  end

  rule if_ 'if' $condition 'then' $texpr [(otherwise | 'else') $fexpr]
    if def? $fexpr then
      expr '$condition ==> $texpr && !$condition ==> $fexpr'
    else
      expr '$condition ==> $texpr'
    end
  end

  rule sum the $sum is the sum of $input1 and $input2
    expr '$sum == $input1 + $input2'
  end

  rule pattern_valid exactly one $x in patterns matches string $y
    fail 'Exist exactly one construcs are unsupported'
  end
\end{lstlisting}

A transformation rule has two main components: a pattern which a given
string should match and a body expression which defines how the
text should be transformed (see Listing~\ref{lst:ruleexample}).
To allow cross-referencing between rules, each rule must additionally be
given a unique identifier.

\paragraph{Patterns}
\label{par:Patterns}

To find the matching transformation for a given string description
(written in the behavior in either a requiring block or an ensuring block), we allow the
definition of patterns which describes the appearance of the string.
A pattern is composed of a series of exact strings, or variables,
denoted by the sigil `\$', which represents a given word.
Furthermore to allow greater re-usability we also allow alternation between
series of sub-patterns (see Listing~\ref{lst:ruleexample}) surrounded by parentheses and
separated with a vertical bar `$\vert$' symbol, and we allow optional parts which are
surrounded with brackets.  This is similar to how regular regular
expressions work except Kleene-star `*' construction is not supported
resulting in that a sub-pattern can be matched at most once.

\paragraph{Transformation Expressions}
\label{par:Transformation Expressions}
While patterns specify how strings are matched, transformation expressions
specify how to translate a given pattern to a formal logical expression.
In our language we support four kinds of transformation expressions:
generation, forwarding, conditional and failure (see Listing~\ref{lst:ruleexample}).

Generation is written with the keyword `\emph{expr}'
followed by the string which should be generated on output, where variables
in that string are replaced by the words captured in the pattern.

Forwarding enables a rule to refer to previously written rules that should further handle the
transformations and thus enhance re-usability. A forward expression is written with the keyword `\emph{match}',
followed by the name of the rule that should handle further transformations,
followed by a series of assignments to the variables of the forwarded
rules' patterns.
To handle variables that occur in alternations and optional parts of a
pattern, the transformation permits the usage of if def?-then-else expressions.
These expressions allow to check if a series of variables exist,
and handle transformation flow accordingly.
In the end failures, denoted by the keyword `fail',
are useful as placeholders until an actual transformation is written.

\subsubsection{Evaluation of Transformation Rules}
\label{sub:Evaluation of Transformation Rules}

The process of evaluating a feature description using the transformation rules require two steps:
\begin{enumerate}
  \item Pattern-match the given string to find the rule to evaluate
  \item Executing the body of the rule with the captured variables
\end{enumerate}
To accommodate the first part of the process we translate each pattern to a
similar named regular expression. As an example, take the pattern string of rule positive from Listing~\ref{lst:ruleexample}, it will be translated 
from \textit{\$input is a positive number} to \textit{\textbackslash s*+(?\textless input\textgreater\textbackslash S+)\textbackslash s* is a positive number}.

Then we try to match the string with each pattern until we succeed,
and then each variable in the pattern is captured into an environment
which maps variable names to values.
After a match is found, the second process is started where the body
for the matched pattern is evaluated given the captured environment of assigned variables.
The evaluation is dependent on which type of expression is specified
and is done as follows:

\begin{description}
  \item[Generation] The generation statements are evaluated simply
    by replacing all variables in the desired result expression with their 
    value in the environment. 
  \item[Forwarding] Transformations that use forwarding are evaluated by
    assigning each of the variables in the pattern with the values in the
    current environment and then further evaluating the forwarded rule.
  \item[Conditional] To evaluate a conditional expression, we check if all
    the variables specified exist in the current environment.
    If they exist the `then' inner transformation is evaluated recursively
    with the given variables, otherwise the `else' inner transformation
    is evaluated similarly.
  \item[Failure] We simply throw an exception with the given message
    if a failure statement is met.
\end{description}

\subsubsection{Transformations as an Opportunity of Reusing Expressions}
\label{ssub:TransformationsasanOppurtunityofReusingExpressions}
As a consequence of having the transformation language, Lettuce,
we allow programmers to generate templates (e.g. Java classes with contracted methods and empty bodies) from behavior descriptions in a flexible way. This allows semi-automatizing of routine tasks by leveraging model-driven development, thus possibly saving developer time and effort.


\subsection{Tomato -- a JML-compatible Language for the Generalized Contracted Software Model}
\label{sub:Tomato-aJML-compatibleLanguagefortheGeneralizedContractedSoftwareModel}


To represent contracted software we have chosen to create a subset of JML called Tomato,
which focuses primarily on the interface-specific parts of Java.
Tomato is a specific DSL implementation of our general programming-by-contract output model, and thus has the same set of features
namely preconditions, postconditions and invariants \footnote{the Tomato language supports invariants, but the behaviors do not}.

\subsubsection{Representing Logical Expressions}
\label{sub:Representing Logical Expressions}
To allow the same type of expressiveness as in JML, we support all pure
logical expressions, i.e. expressions that do not have any side effects and evaluate to a boolean.
Furthermore, Tomato similarly to JML \cite{leavens2006design},
support extensions that allow formulation of more complicated
expressions such as implications and quantifiers,
and those that specify change in the state of an object such as \textbackslash result
and \textbackslash old($\cdot$).

\begin{lstlisting}[language=Java,caption={Contracted Method for Adding Natural Numbers},label={lst:jmlexample}]
  class Calculator {
    //@ requires input_1 > 0
    //@ requires input_2 > 0
    //@ ensures \result == input_1 + input_2
    //@ ensures \result > 0
    public /*@ pure @*/ int add(int input_1, int input_2) { ... }
  }
\end{lstlisting}

\subsubsection{Preconditions and Postconditions}
\label{sub:Preconditions and Postconditions}

As the idea behind code contracts is the ability to specify the behavior of
an object, Tomato allows specification of pre- and postconditions of methods.
Preconditions specify which prerequisites the caller of a method must
satisfy in order for the method to succeed and fulfill its postcondition.
Preconditions in Tomato are declared using the \emph{`requires'}
keyword (see Listing~\ref{lst:jmlexample}).
Conversely, postconditions, declared with the \emph{`ensures'} keyword,
specify which guarantees the callee of a method will give after an invocation.

\subsubsection{Providing a Balanced Solution for Contracted Software}
\label{ssub:ProvidingaBalancedSolutionforContractedSoftware}
Although Tomato does not support the full suite of constructs supported by JML, it provides a clean and useful way of contracting software. Because of 
the simple model it is based on, further development of the language can easily be made to better reflect the capabilities of JML.
