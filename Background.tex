\section{Backround}

\subsection{Behavior Driven Development}
\label{sub:bdd}
Behavior Driven Development is a fairly recent software development process. It is based on Test Driven Developement (TDD) and developed by Dan North as a response to his own annoyance over occuring repeatedly misunderstandings when using TDD because of not knowing the behavior of the software as wished by stakeholders until late in the development process \cite{north2006}.

The main component of BDD is called a feature. 
A feature is a part of a software system,
which can be specified by a range of specific behaviors called scenarios.
Such scenarios can be either specific consisting solely of a range of
assumptions and assertions, or be outlined in a way that allows
specification of a range of different example inputs.
Moreover feature descriptions allow specification of a background clause,
as a way to set the general context for the specific scenarios.

Listing \ref{lstbddfeature} shows a simple example of a feature that describes the scenario of addition. 


\begin{lstlisting}[caption={Sample Addition Feature},label={lst:bddfeature}]
Feature: Adding
  Scenario: Add two numbers
    Given the input "2+2"
    When the calculator is run 
    Then the output should be "4"
\end{lstlisting}

If the reader is further interested in basic feature definitions,
please refer to “The Cucumber Book” \cite{hellesoy2012}.

\subsection{Tools}