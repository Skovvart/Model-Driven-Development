\section{Background}

\subsection{Behavior Driven Development}
\label{sub:bdd}
Behavior Driven Development is a fairly recent software development process. 
It is based on Test Driven Development (TDD) and developed by Dan North as a response to his own annoyance over repeated misunderstandings when using TDD because of not knowing the behavior of the software as wished by stakeholders until late in the development process \cite{north2006}. 
The main goal of BDD is to allow for better communication and collaboration between stakeholders and developers through specifying behaviors from which tests can be automatically generated. 

The main component of BDD is called a feature. 
A feature is a part of a software system,
which can be specified by a range of specific behaviors called scenarios.
Such scenarios can be either specific consisting solely of a range of
assumptions and assertions, or be outlined in a way that allows
specification of a range of different example inputs.
Moreover feature descriptions allow specification of a background clause,
as a way to set the general context for the specific scenarios.

Listing \ref{lst:bddfeature} shows a simple example of a feature that describes the scenario of addition. XXX continue example XXX.

\begin{lstlisting}[caption={Sample Addition Feature},label={lst:bddfeature}]
Feature: Adding
  Scenario: Add two numbers
    Given the input "2+2"
    When the calculator is run 
    Then the output should be "4"
\end{lstlisting}

If the reader is further interested in basic feature definitions,
please refer to “The Cucumber Book” \cite{hellesoy2012}.

\subsection{A Generalized Model for Contracted Software}
\label{sub:AGeneralizedModelforContractedSoftware}

\subsection{Tools}
The implementation of our project is done using the Eclipse Modelling Framework (EMF).
The framework permits the use of model-driven development in a complete sense from designing models to implementing transformations, and makes
it easier to develop reusable domain specific languages (DSL) and software. Specifically we are using four parts of the framework:
\begin{description}
\item[Ecore] Meta-modelling language which allows the description of software models and other meta-models. Given an Ecore model, the EMF can
  generate a complete set of tools for use with it, such as model code, test cases and eclipse editor tools. This utilizes the strengths
  of model-driven development -- specifically modularity and re-usability, and thereby allowing the extension of Eclipse itself with small effort.
\item[Xtext] EMF also includes a method of developing specialised DSLs with the Xtext framework. By simply specifying an LL grammar and linking to an Ecore model
  Xtext does not only provide a parser, but also Eclipse extensions for syntax highlighting, auto-completion and serialization.
  Xtext allows a single model to be reused in multiple DSLs. This is something which we specifically utilize in our framework,
  in which by simply changing the serializer we can support other languages than JML.
\item[Xtend] To transform our behaviors to contracts (see Section~\ref{sub:Evaluation of Transformation Rules}) we utilize Xtend.
  Xtend augments Java with syntactic and declarative extensions, thereby allowing Model-to-Model transformations using a higher level of abstraction.
\item[MWE2] The Model Workflow Engine can be used to automate transformations between models and code generation, by utilizing a simple plug-in based
  workflow model. In our framework we have combined built-in and custom made components into a simple yet powerful workflow, which handles every detail
  from parsing our behavior models, to generating code for contracted software.
\end{description}
