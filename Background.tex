\section{Background}
\label{sec:Background}

\subsection{Behavior Driven Development}
\label{sub:bdd}
Behavior Driven Development is a fairly recent software development process, based on Test Driven Development (TDD).

The main component of BDD is called a feature. 
A feature is a part of a software system,
which can be specified by a range of specific behaviors called scenarios. Scenarios can be either specific, consisting solely of a range of assumptions and assertions, or be outlined in a way that allows
specification of a range of different example inputs.
Moreover feature descriptions allow specification of a background clause,
as a way to set the general context for the specific scenarios.

Listing \ref{lst:bddfeature} shows a simple example of a feature that describes the scenario of addition. 
The title, `\emph{Adding}', is followed by the summary and a scenario. The keywords in the scenario definition `\emph{Given}', `\emph{When}' and `\emph{Then}' respectively specifies the needed state, the action performed and the outcome of the scenario. 
Several `\emph{Given}', `\emph{When}' and `\emph{Then}' blocks are concatenated with the keyword `\emph{And}'.

\begin{lstlisting}[caption={Sample Addition Feature},label={lst:bddfeature}]
Feature: Adding
 In order to avoid mistakes 
 As an accountant
 I want to be told the sum of two numbers

   Scenario: Add two numbers
     Given the input "2+2"
     When the calculator is run 
     Then the output should be "4"
\end{lstlisting}

Listing \ref{lst:cucumberoutput} shows the output of running Listing \ref{lst:bddfeature} through a BDD-framework like Cucumber.
These are called `\emph{step definitions}' and from these, automated tests can be generated. 
Note: The standard language in Cucumber is Ruby, however, as an open source tool, the number of supported languages are constantly growing. 

\begin{lstlisting}[caption={Sample Cucumber Output},label={lst:cucumberoutput}]
Given /^the input "([^"]*)"$/ do |arg1|
   pending
end
When /^the calculator is run$/ do
   pending
end
Then /^the output should be "([^"]*)"$/ do |arg1|
   pending
end
\end{lstlisting}

As mentioned in the introduction, instead of generating tests from features, we want to be able to generate contracts. 
If the reader is further interested in feature definitions, please refer to “The Cucumber Book” \cite{hellesoy2012}.


\subsection{Tools}
The implementation of our project is done using the Eclipse Modelling Framework (EMF).
The framework permits the use of model-driven development in a complete sense from designing models to implementing transformations, and makes
it easier to develop reusable domain specific languages (DSL) and software. Specifically we are using four parts of the framework:
\begin{description}
\item[Ecore] Meta-modelling language which allows the description of software models and meta-models. Given an Ecore model, EMF can
  generate a complete set of tools for use within it, such as model code, test cases and Eclipse editor tools. This utilizes the strengths
  of model-driven development -- specifically modularity and re-usability, thereby allowing the extension of Eclipse itself with small efforts.
\item[Xtext] EMF also includes a method of developing specialised DSLs called Xtext. By simply specifying an LL grammar and linking to an Ecore model,
  Xtext does not only provide a parser, but also Eclipse extensions for syntax highlighting, auto-completion and serialization.
  Furthermore, Xtext allows a single model to be reused in multiple DSLs. This is something we specifically utilize in our framework,
  in which we can support other languages than JML by simply changing the serializer.
\item[Xtend] To transform our behaviors to contracts (see Section~\ref{sub:Evaluation of Transformation Rules}) we use Xtend.
  Xtend is a programming language that augments Java with syntactic and declarative extensions, thereby allowing Model-to-Model transformations using a higher level of abstraction.
\item[MWE2] The Model Workflow Engine can be used for code generation and to automate transformations between models, by utilizing a simple plug-in based
  workflow model. In our framework we have combined built-in and custom made components in a simple yet powerful workflow, that handles every detail
  from parsing our behavior models, to generating code for contracted software.
\end{description}
